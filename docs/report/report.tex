\documentclass[11pt]{amsart}
%\documentclass[a4paper,12pt,oneside]{article}
\usepackage[english]{babel}
\usepackage[utf8]{inputenc}
\usepackage[T1]{fontenc}
\usepackage{listings}
\usepackage{amsmath}
\usepackage{amsthm}
\usepackage{float}
\usepackage{url}
\usepackage{verbatim}

\title{DD2488 - Compiler Construction\\Project Report}
\author{Samuel Wejeus \and Emil Arfvidsson}

\begin{document}
\maketitle

\section{Introduction}
A compiler is a complicated piece of machinery, a computer program, designed to transform high level source code written in some programming language into some lower level language understandable, and runnable on various target architectures i.e. different types of processors.

This report presents the construction, and inner workings, of a compiler for the high level programming language \textit{MiniJava} (which is a subset of the Java programming language) created as a project in the course \textit{DD2488 - Compiler Construction}\cite{appel}. We named the project \textbf{Cortado} which is a small and strong type of italian coffee.\footnote{Wordplay is another of our amusements besides building compilers}

The report will present the various tools used to facilitate the construction of the compiler and explain how the different stages of compilation have been implemented. The intended reader is a person who has rudimentary understanding of the workings of a compiler and our goal is to give the reader insight and a basic understanding of the inner workings and design choices we made for our implementation of MiniJava. This report is also intended as documentation for ourselves on things we found difficult and their solution.

\section{Overview}
During the course of compilation a compiler goes through several stages where the input (i.e. the file of MiniJava source code) is tested and/or transformed into various representations (most common: a tree structure in various forms). This is to make the compiler more modular and easier to maintain, an input at some stage is usually the output from the previous stage. On the whole, the entire process of transforming input source code to an executable file is separated into the following stages:

\begin{itemize}
\item Lexicographical analysis
\item Parsing
\item Type and scope checking
\item Translation to intermediate code
\item Code generation
\item{Liveness}
\item{Register allocation}
\item Assembly and linking (not covered)
\end{itemize}

As a reference and as an insight into our project each step will be explained briefly and focus is towards how we covered each stage in our implementation. The last step of compilation \textit{Assembly and linking} was not part of this project and have been excluded from the compiler. To complete the process of creating a real exactable binary we here rely on ordinary assemblers such as \textit{nasm}\cite{nasm} to preform the last step. We note that upon successful compilation using our compiler a complete, syntactically correct, assembler source file will be given as output.

The project of creating a compiler for the MiniJava programming language (hereby just referred to as 'MiniJava') was implemented using the ordinary Java programming language. 

\subsection{System Requirements}
For running a precompiled binary of our complier the only requirement is a runtime environment for Java (JRE). Requirements for building the source code the following additional tools and libraries are needed. The MiniJava compiler have been successfully built with the versions of relevant libraries as noted below but more recent versions should work to.

\begin{itemize}
\item Java-cup-11a
\item JFlex 1.4.3
\item JUnit 4.10 (only needed for test cases)
\end{itemize}

To build the project, extract the project package and type \texttt{ant} in the root folder, this creates a jar file containing the compiler. To compile your own MiniJava source code files type \texttt{./cortado your-source-file}. This will perform all stages of compiling and output .j files of Jasmin code in the directory \texttt{cortadoout}.

\subsection{Architecture and MiniJava Extensions}
Our main target is the JVM, which outputs Jasmin code. Initially our target target architecture was AMD64 assembler, but due to lack of time this target was never completed. Since MiniJava is a limited subset of Java many operators and functionality have been exclude such as for loop and much of the boolean logic. As a bonus in the DD2488 course we were encouraged to implement some of this missing functionality as extensions. None of these have been implemented.

\section{Lexing and Parsing}
The first phase of the compiler consists of transforming the input source into tokens and parsing these against the grammar for MiniJava to discover possible syntax errors. At this stage we utilize two different tools: JFlex and JavaCUP. 

JFLex is a scanner generator that scans a text file and transforms input, matched against patterns written using regular expressions, to tokens (also called \textit{lexing}). This is the first point of error checking, if any invalid MiniJava characters or strings are encountered the compilation is halted.

A natural companion to JFlex is JavaCUP (Constructor of Useful Parsers) which is a parser generator that creates LALR parsers\cite{appel}.

Both JFlex and JavaCUP works as follows: each tool takes an input file consisting of instructions, specified according to the syntax of the tool, of how to perform its actions. Running the tool in question on the corresponding file generates a new Java \textit{source code file} that is later included and instantiated as an object in the framework for the compiler. 

Though learning how to use both JFlex and JavaCUP can be a bit tricky in the beginning the benefit gained is \textit{much} easier maintenance and development since the alternative, building the automatons\cite{automata} yourself, is considered way more advanced.

The output from the lexing and parsing phase is a tree created and linked together by various subclasses in the \texttt{syntaxtree} package. These correspond to a top-level object called \textit{Program} consisting of \textit{Classes} which in turn consists of \textit{Methods} and so on corresponding to the grammar given for MiniJava\cite{grammar}.

Input specifications for lexing and parsing resides in the \texttt{src/lexer} and \texttt{src/parser} folders of the project and output is generated in \texttt{gen/}. Classes used to create the syntax tree is located in the \texttt{syntaxtree} package.

\section{Scope and Type Checking}
	If the parser is successful, it returns the abstract syntax tree (\textit{AST}) representing the program. The next step is to verify that variables are declared before use and that the types of variables are consistent with what they are assigned. Scope and type checking is done as a two pass procedure. The first pass reads the classes and methods of each class. A representation containing information about each class and method is saved in the classes \texttt{ClassScope} and \texttt{MethodScope}. In each \texttt{MethodScope} the list of parameters and their type is saved. The local variables are also saved. The first pass is now done.

The second pass is implemented in the class \texttt{SlowTypeVisitor}. It verifies that the types in assignments, method calls and expressions are correct. If the left hand side of an assignment has type \textit{A}, the \texttt{SlowTypeVisitor} verifies that the right hand side is also of type \textit{A}. This is done using a visitor pattern, where the type of a node is returned in each visit. Figure ~\ref{fig:timestypecheck} illustrates an example for how this type checking is done in a typical node in the tree.

	\begin{figure}[H]
		\begin{verbatim}
Type visit(Times node) {
    if not node.e1.accept() is of type Integer
        error "Left hand side not of type Integer"
    if not node.e2.accept() is of type Integer
        error "Right hand side not of type Integer"

    return IntegerType
}
		\end{verbatim}
		\caption{Simplified visit example in JasminVisitor}
		\label{fig:timestypecheck}
	\end{figure}

\section{Translation to Jasmin}
	One of the MiniJava compiler backend targets is the JVM. The JVM target requires Jasmin\cite{jasmin} output which is then compiled into .class files that Java can run. The MiniJava compiler implements the Jasmin code generation in a single class called \texttt{JasminVisitor}. It outputs one \texttt{.j} file for each class in the input file.

The whole program is at this point represented in the form of an abstract syntax tree (AST). The \texttt{JasminVisitor} works by visiting the AST in a depth-first fashion. Each visit of a node outputs the corresponding Jasmin code. Visiting a node expects that to leave the calculated value (if any) on the stack. An example of visiting a \texttt{Times} node is shown in figure ~\ref{fig:timesexample}. The \texttt{Times} node has a left and a right expression. After the left and right have been visited their resulting values are expected to be found on the stack. The instruction \texttt{imul} multiplies the two top elements on the stack and replaces them with the result.

	\begin{figure}[H]
		\begin{verbatim}
		visit(Times node) {
		    visit(node.e1);
		    visit(node.e2);
		    write("imul");
		}
		\end{verbatim}
		\caption{Simplified visit example in JasminVisitor}
		\label{fig:timesexample}
	\end{figure}

The JVM requires a minimal stack depth to be specified for each method and thus the Jasmin interpreter also requires a stack depth to be specified per method. This depth is calculated while visiting the nodes. Each visit method returns the depth required to perform the calculation. The two visit calls in the \texttt{Times} example in figure ~\ref{fig:timesexample} return their required depth. The depth of the \texttt{Times} visit itself is then the maximal value of the two depths -- plus the depth required to store the intermediate value. 

\section{Translation to Intermediate Code}
All stages up to, and partly including, the stage of intermediate code generation is commonly referred to as the \textit{front end}, the stages following current stages is referred to as the \textit{back end}, intermediate code is what separates the two. Intermediate code (IR) is a sort of pseudo assembler and used to make the translation several different architectures easier. 

What we do during this stage is to transform the incoming program being compiled (now in the form of a tree where scope and type errors should be non existent) and transform it into a \textit{linear structure}. This new structure should mimic that of a real processor which means that all abstract constructions, such as: \texttt{int x = 3+4;}, have to be converted into a series of instructions of \textit{how} a processor actually would carry out the, in this case, addition.

To implement this procedure we have utilized an additional structure called a \textit{Translator}. The reason for this, and what the translator does, is due to that some constructs, such as conditional statements, need to be translated using several \texttt{jump} instructions. It is possible at later stages of the compilation to detect that some of these \texttt{jump}'s is unnecessary or could be rearranged in a more efficient order depending on if the statement need to return a value of not. Using a translator we can instead create a 'meta' statement and at a later stage when the context for this statement is known the translator can easily convert the statement in a efficient way by optimizing it to a, for instance, statement that returns a value. The translator is located in the \texttt{se.cortado.ir.translate} package.

\section{Code generation}
There's now an intermediate tree representation of the program. Each node in this tree is simpler compared to the nodes in the \textit{AST}. The nodes also correspond more closely to instruction commonly found in CPUs, which makes it possible at this point to generate assembler for the program.

The class \texttt{Codegen} is an implementation of Maximal munch\cite{appel} algorithm. It recursively visits all the nodes in the IR tree and outputs instructions for a set of nodes at a time. It tries to minimize the amount of instructions and registers used by 'tiling' the tree. Each tile corresponds to one instruction. Certain assembler instructions can perform enough calculations to cover several nodes, meaning that there is not necessarily exactly one instruction per node in the IR tree. The Maximal munch algorithm is defined as greedily covering as many nodes as possible per instruction, starting from the root node. The IR tree nodes are either subclasses of  \texttt{IR\_Stm} or \texttt{IR\_Exp}. In \texttt{Codegen} there are two recursive methods called \texttt{munchStm} and \texttt{munchExp}, that consumes the nodes. \texttt{munchExp} returns a \texttt{Temp}. Each \texttt{Temp} represents a register. At this point there is no limit to how many registers are used.

When munching, instructions are emitted and stored in a list. Each instruction is a subclass of \texttt{Instr}. An instruction is made up partly by the actual assembler instruction, but also a list of \texttt{Temp}'s used and defined in the instruction. The used \texttt{Temp}'s are the ones that are read by this instruction. The defined \texttt{Temp}'s are the temporaries that are written by this instruction. The used and defined values are used later in the analysis of the program.

\section{Liveness and Register Allocation}
Through out the compilation process so far virtual temporaries have been used to hold variable. Since on a real CPU all calculations have to be performed on variables located either in memory, or preferably, in registers we need to link these temporaries to real registers, the problem is that the number of registers is limited and we need to determine at any point in time which variables is currently needed and need to be placed in registers.

This problem is solved using \textit{flow control}. When flow control starts the input for this stages is a linear list of \texttt{Assem} instructions. The flow analysis is done in two parts: First, the control flow of the \texttt{Assem} program is analyzed by traversing the statements backwards, producing a control-flow graph. The control-flow graph is simply a graph representation of \textit{"how"} the \texttt{Assem} program executes and consists of nodes corresponding to each statement (or instruction) and directed edges between them modeling the flow of the program.

Second part is what is called \textit{liveness analysis}. Liveness analysis could be used for various kinds of optimizations but here our goal is to focus on what set of variable is live at the same time. What liveness now does is to traverse the control-graph determining at each node the set of variables that needs to be \textit{live} at that point so that future points that potentially use those variables can be guaranteed correct data. The mathematical definition for liveness at a node is listen below:

\begin{align*}
in[n] &= use[n] \cup (out[n] - def[n]) \\
out[n] &= \bigcup_{s \in succ[n]} in[s]
\end{align*}

Here $in[n]/out[n]$ is edges, $use[n] =$ a right hand side \textit{usage}, $def[n]$ is a variable definition and $succ[n]$ is the successor node of $n$. The equation given above is executed for each node and the algorithm terminates when, for a node, both its in and out edges satisfy the equation.

The result of this procedure is an \textit{interference graph} which is a graph consisting of the variables discovered in the control-flow graph and two nodes (i.e. variables) are connected if, and only if, they are \textit{live} at the same time. The reader should note that this graph could, and most likely will, be disconnected.

\subsection{Graph coloring}
During the finalization of the interference graph it is still assumed that the number of registers is unlimited, what we done so far is simply to determine the minimal subset of variables than \textit{must} be live at each point in time. For this we use the \textit{graph coloring} procedure evaluated with \textit{K} colors, where \textit{K} is the number of registers available. If we allocate the registers according to the graph coloring scheme we will be guaranteed that variables that are live at the same time never will interfere with each other by, possibly, overwrite each others values.

\section{Conclusions}
If one where to compare the development efforts required to implement the different backends (here comparing x86 with JVM implemented through Jasmin) we can conclude that creating an executable using Jasmin is several magnitudes easier than creating an backend for x86. This is due to several reasons.

First regarding the literature used, the book we followed to complete the project, \textit{Modern Compiler Implementation in Java, 2nd edition} by Appel\cite{appel} is bad. Lets just get that out of our system. Appel tries too much without actually explaining anything with enough detail to actually be usable. The criticism is targeted towards the chapters about: IR code translation and code generation.

A big difference between a JVM backend and x86 is an additional constraint that comes with generating x86 code namely, register allocation and the requirement for a perfect linear program. Using Jasmin you can completely ignore those constraints and just emit code for each method separately and let the Jasmin translator care about that the code block are ordered in an executable fashion (i.e. creating \textit{traces}). Since Jasmin uses a stack, one per method, you do not have to bother about register allocation and thereby the translation of temporaries used to hold variables is straight forward, just push them on the stack.

On the whole a compiler is a very complicated piece of machinery but due to strong modularization into different stages it is fairly easy to get a good understanding of the totality by considering one module at a time. Good theory lead to good programs. For the front end of the compiler the theories used (for lexing, parsing) can be considered sound and implementation is easy. For the backend the theories, as explained in the course literature, can not be considered complete and an analogy suitable here is "stumbling a bit in the dark".

The negative side of heavy modularization is that you get a bigger code base. More code is not always a bad thing and we consider the tradeoff between the two in favor for modularization because it gives a far better understanding of the inner workings of the compiler which we deem most important as it makes maintenance and verification easier, both important parts due to the requirement for (hopefully) absolute correctness.

\begin{thebibliography}{9}

\bibitem{appel}
	Andrew W. Appel,
	\emph{Modern Compiler Implementation in Java, 2nd edition}.

\bibitem{nasm}
	Nasm OpenSource Project,
	\emph{The Netwide Assembler}.
	\url{http://www.nasm.us/}

\bibitem{jasmin}
	Jasmin,
	\emph{Jasmin}.
	\url{http://jasmin.sourceforge.net/}

\bibitem{automata}
	\emph{Automata Theory}.
	\url{http://en.wikipedia.org/wiki/Automata_theory}

\bibitem{grammar}
	\emph{MiniJava Grammar}.
	\url{http://www.csc.kth.se/utbildning/kth/kurser/DD2488/komp12/project/newgrammar.pdf}

\end{thebibliography}
\end{document}  